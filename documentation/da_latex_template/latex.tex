\chapter{Kurze Einf"uhrung in \LaTeX}
    F"ur die Diplomarbeit bitte dieses Kapitel im Masterdokument
    einfach auskommentieren (\%~davor).

	\LaTeX{} ist eine sehr leistungsf"ahige Software zum Verfassen wissenschaftlicher Arbeiten.

	Wichtige \LaTeX{}-Infos findet man in \cite{l2kurz}.

	\section{Gliederung (ich bin eine "Uberschrift -- section)}
	Ein neues Kapitel beginnt mit dem \verb|\chapter {Kapitelname}|-Befehl, z.\,B. die Kapitelüberschrift auf dieser Seite (Kurze Einführung \dots).

	Weitere Gliederungen sind:\\
	\verb|\section{Titel}|, \verb|\subsection{Titel}|,
	\verb|\subsubsection{Titel}|

	Das sieht dann so aus:


	\subsection{Ich bin eine Unter"uberschrift -- subsection}

		\subsubsection{Ich bin eine Unterunter"uberschrift -- subsubsection}

		Ein Inhaltsverzeichnis wird mit \verb|\tableofcontents| eingef"ugt und ber"ucksichtigt alle Kapitel, "Uberschriften, Unter"uberschriften usw.

	\subsection{Aufz"ahlungen}
		Aufz"ahlungen k"onnen ohne Nummern nur mit Bullets ausgef"uhrt sein, das macht man mit \verb|\itemize|. In eckigen Klammern nach \verb|\item| kann man statt der Bullets alternative Symbole verwenden:

		\begin{itemize}
			\item Franzi
			\item[\smiley] Peppi
			\item[\frownie] Ferdl
		\end{itemize}





		 Soll die Aufz"ahlung auch nummeriert sein, erfolgt das mit \verb|enumerate|.

		\begin{enumerate}
			\item Franzi  \newline
			      Lois
			\item Peppi
			\item Ferdl
		\end{enumerate}

		\pagebreak
		\subsection{Tabellen}
		Tabellen sind ein Bischen haarig\dots

		Hier bitte \cite{l2kurz} durchlesen und oder deb Tabellenasistenten bei \TeX Studio verwenden.


		\subsection{Dies und das}
		Ein Leerzeichen das nie umgebrochen wird, macht man mit einer Tilde \verb|~|, also z.\,B. \verb|auf S.~\pageref{fig:HTL_Logo_neu}|. Da sieht man auch gleich, wie man auf eine Seitennummer verweist.

		Ein Zeilenumbruch erfolgt mit \verb|\\| oder \verb|\newline|. \verb|\newline| geht immer, \verb|\\| fast immer.

		Eine Zeile im Quelltext zwischen zwei Zeilen macht einen neuen Absatz.

		Ein Seitenumbruch erfolgt mit \verb|\pagebreak|.

		Will man auf einer Seite doch noch was unterbringen was \LaTeX aber schon auf die n"achste Seite bringen will, kann man die aktuelle Seite  mit \verb|\enlargethispage| vergr"o"sern. Danach hat sich ein manueller Seitenumbruch bew\"ahrt.

		Wenn man ein Literaturzitat einfügen will, erfolgt das mit dem \verb|\cite|-Befehl.	Dazu muss man im  File \verb|literaturverzeichnis.tex| der Vorlage entsprechend das Zitat einf"ugen. Man kann das mit der freien Software \verb|JabRef| oder mit der Webapplikation \verb|Mendeley| auch professioneller l"osen.



		\section{Abbildungen}
		\label{kap:abbildungen}
		Eine Graphik setzt sich aus zwei Teilen zusammen:

		\begin{itemize}
			\item Umgebung \verb|figure|\\
			      Diese stellt Nummer der Abbildung, Bildunterschrift und Referenz f"ur Verweise zur Verf"ugung
			\item Der eigentlichen Abbildung, meist eingef"ugt mit \verb|\includegraphics| (pdf, png, jpg)
		\end{itemize}

		Ein Beispiel daf"ur ist Abb.~\ref{fig:HTL_Logo_neu} auf S.~\pageref{fig:HTL_Logo_neu}.






		\begin{figure}[H]
			\centering
			\includegraphics[width=70mm]{htlhl_logo.png}
			%\caption[]{}
			%\caption{Das neue und unheimlich \fbox{xx zensuriert xx} Schullogo}
            \caption{Das neue Schullogo \cite{htl_logo}}
			\label{fig:HTL_Logo_neu}
		\end{figure}






		Die Option \verb|[H]| in Kombination mit dem Package float (in der Pr"aambel: \verb|\usepackage{float}|) l"asst die Abbildung genau an der Stelle stehen, wo sie eingef"ugt wurde, das wollen wir so!

		Will man in \LaTeX{} in eine Abbildung was reinschreiben, hilft das Paket \verb|overpic|, in der Pr"aambel muss dann stehen \verb|\usepackage{overpic}|. Erkl"arungen findet man bei Tante Google, das hat sich z.\,B. bei Pfeilen und dergl in Abbildungen bew"ahrt.


		\pagebreak
f		\section{Mathematikmodus}

		Der Winkel~$\alpha$.

		Die Wurstsemmel kostet 20\,\$

		L"angere Gleichungen f"ugt man in einer \verb|equation|~Umgebung ein:

		\begin{equation}
			\pm\sqrt{a^2 + b^2}=c
			\label{equ:pythagoras}
		\end{equation}

		Sie finden den Lehrsatz des alten Griechen in Glg.~\ref{equ:pythagoras} auf S.~\pageref{equ:pythagoras}.

		Will man die Nummer bei der Gleichung nicht, nimmt man die \verb|equation*|~Umgebung:

		\begin{equation*}
			\tan{\alpha}=\frac{GK}{AK}
		\end{equation*}

		\subsection{SI-Einheiten}
			Es gibt zwei leider nicht kompatible Pakete für Einheiten. Das nicht verwendete muss in der Präambel auskommentiert werden. Wir verwenden das neuere Paket \verb|siunitx|

			\subsubsection{siunitx}
				 Damit Zahl, Abstand, Einheit, Mathematikmodus oder nicht und Schrifttype passen, wird am besten das Paket \verb|siunitx| verwendet.

				 $\SI{2.63}{m/s} $ geschreiben als \verb|$\SI{2.63}{m/s}$|, \\
				 \SI{2.63}{\meter / \second}  geschrieben als\verb|\SI{2.63}{\meter / \second}|, \\
				 \SI{2.63}{\meter\per\second} geschrieben als \verb|\SI{2.63}{\meter\per\second}|

				Die Befehle \verb|\metre| statt \verb|m| sind besonders bei zusammengesetzten Einheiten sinnvoll, z.\,B.

				\SI{23}{\micro\metre} geschrieben als   \verb|\SI{23}{\micro\metre}|

				Die Temperatur beträgt \SI{23}{\celsius} geschrieben als   \verb|\SI{23}{\celsius}|

				Der Winkel beträgt \SI{23}{\degree} geschrieben als   \verb|\SI{23}{\degree}|, besser allerdings funktioniert\\
				Der Winkel beträgt \ang{23,5} geschrieben als   \verb|\ang{23,5}|,\\
				Der Winkel beträgt \ang{23;30;3} geschrieben als   \verb|\ang{23;30;3}|


				Alternativ kann z.\,B. statt \verb|SI{23}{\degree}| auch \verb|\unit{23}{\degree}| geschrieben werden.

				Die Kosten betragen 23,60\,\euro{} geschrieben als \verb|23,60\,\euro{}| hat nichts mit dem \verb|siunitx|-Paket zu tun.




		\pagebreak
		\section{Literaturverzeichnis}
			\LaTeX verfügt über eine sehr gute Möglichkeit, Literaturverzeichnisse zu erstellen. Dazu wird ein sogenantes \verb|*.bib|-File angelegt, in dem die verwendete Literatur angeführt wird (es können in diesem File auch nicht verwendete Werke stehen).

			Bekannte Werke kann man z.\,B. mit der der Literatursuchmaschine von Mendeley suchen und mit dem Mendeley Reference Manager (dieses Programm muss am PC installiert werden) in einem  ein \verb|*.bib|-File exportieren.

			Das mit einigen Bespielen schon angelegt File heißt \verb|my_library.bib|.

			Den allgemeinen Aufbau von \verb|*.bib|-Files und kleine Beispiele findet man z.\,B. bei  \cite{wikibook:biber}.

			Viele bunte Bilder zur Technik findet man in \cite{fk_metall}






