\documentclass[a4paper,12pt]{article} % Defines the document class
\usepackage[utf8]{inputenc} % For UTF-8 encoding
\usepackage{amsmath} % For advanced mathematical formulas
\usepackage{graphicx} % For including images
\usepackage{hyperref} % For clickable links

\title{An Example LaTeX Document}
\author{John Doe}
\date{\today}

\begin{document}

\maketitle % Generates the title, author, and date

\tableofcontents % Generates the table of contents

\section{Introduction}
LaTeX is a typesetting system commonly used for technical and scientific documents. It is particularly well-suited for documents with complex formatting requirements, such as mathematical equations, references, and tables.

\subsection{Why Use LaTeX?}
Some advantages of LaTeX include:
\begin{itemize}
    \item Professional-quality typesetting
    \item Support for complex formulas and equations
    \item Easy management of references and bibliographies
\end{itemize}

\section{Mathematical Examples}
Here are some examples of mathematical formatting in LaTeX:
\subsection{Inline Equations}
The quadratic formula is written inline as: \( x = \frac{-b \pm \sqrt{b^2 - 4ac}}{2a} \).

\subsection{Displayed Equations}
Displayed equations appear on their own line:
\[
E = mc^2
\]

\section{Including Figures}
You can include images using the \texttt{graphicx} package. For example:

\begin{figure}[h]
    \centering
    \includegraphics[width=0.5\textwidth]{example-image} % Replace "example-image" with your image file
    \caption{An example image.}
    \label{fig:example}
\end{figure}

\section{Hyperlinks}
You can include clickable links using the \texttt{hyperref} package:
\href{https://www.latex-project.org}{LaTeX Project Website}.

\section{Conclusion}
This document demonstrates the basic structure and features of a LaTeX document. For more details, refer to the official LaTeX documentation.

\end{document}
