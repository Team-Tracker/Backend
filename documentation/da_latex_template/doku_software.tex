\chapter{Dokumentation der Software}
	\section{Quellcode}
		wird mit dem \verb"listings"-Pakte eingef"ugt.
		  		
		Ein wiki-Book: \href{https://en.wikibooks.org/wiki/LaTeX/Source_Code_Listings}{\texttt{https://en.wikibooks.org/wiki/LaTeX/Source\_Code\_Listings}}
		  		
		  		
	\section{Funktionen, Strukturen, Datentypen,  Variablen}
		Hier soll die mit \verb"Doxygen" \cite[S.~3]{wiki:doxygen} erstellte Dokumentation stehen
		  		
		Aus \cite{wiki:doxygen} stammt das einfache Beispiel, dass die Syntax in einem c-Kommentar eigentlich vollständig erläutert:

\begin{verbatim}
 /**
  * \brief  Exemplarische Funktion zur Erklaerung von Doxygen
  *
  *         Diese Funktion gibt den uebergebenen Parameter
  *         auf der Konsole aus.
  *
  * \param	parameter   Auszugebende Parameter
  * \return	            Status-Code
  *
  */
 int funktion(int parameter)
 {
     printf("Parameter: %d", parameter);   

     return 0;
 }
\end{verbatim}		 

	Doxygen erzeugt ein verlinktes \verb|*.html|-File bzw. ein \verb|*.tex|-File mit allen Funktionen, Datentypen, Abh"angigkeiten, das nur mehr mit \verb|\input{name_des_doxyoutputs.tex}| eingef"ugt werden braucht. 
		  		
		  		