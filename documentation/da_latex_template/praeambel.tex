%%%%%%%%%%%%%%%%%%%%%%%%%%%%%%%%%%%%%%%%%%%%%%%%%%%%%%%%%%%%%%%%%%%%%%%%%%%%%%%%%%%%%%%%%%%%%%%%%%%%%%%%%%%%%%%%%%%%%%%%%%%%%%%
%%     LaTeX Präambel für die Dokumentvorlage
%%     für Diplomarbeiten an der HTL Hollabrunn
%%     ---------------------------------------------------------------------
%%     !!  HIER BITTE NICHTS ÄNDERN, außer Sie wissen genau, was Sie tun  !! Wir wissen es
%%%%%%%%%%%%%%%%%%%%%%%%%%%%%%%%%%%%%%%%%%%%%%%%%%%%%%%%%%%%%%%%%%%%%%%%%%%%%%%%%%%%%%%%%%%%%%%%%%%%%%%%%%%%%%%%%%%%%%%%%%%%%%%

\documentclass[a4paper,11pt,DIV15,oneside,normalheadings,headsepline,twoside]{scrreprt}
									% mit chapter, section, subsection, subsubsection
%\documentclass[a4paper,12pt,DIV15,oneside,normalheadings,headsepline]{scrartcl}
									% mit section, subsection, subsubsection
% ------------------------------------------------------------------------------------------------------------

% -----------------PAKETE (löschen/hinzugeben nur, wenn man sich sicher ist) -------------------------------------
\usepackage[utf8]{inputenc}		    % damit die Tastatur richtig erkannt wird (passt immer, egal welches OS)
\usepackage{lmodern}
\usepackage[T1]{fontenc} 			% europäische Schrifttypen, wichtig für Copy&Paste aus pdf
\usepackage{textcomp}				% Sonderzeichen wie in l2kurz , S.45
\usepackage{color}

\usepackage[austrian]{babel}		% Abteilungen usw. \usepackage{german} beisst sich mit gnuplottex
\usepackage[style=trad-unsrt, backend=biber]{biblatex}
\usepackage{hyperref}
\definecolor{darkblue}{rgb}{0,0,0.845}
\hypersetup{colorlinks=true, citecolor=darkblue, urlcolor=darkblue, linkcolor=darkblue}
\usepackage{url}
	% macht die URL im Literaturverzeichnis hübscher
	\DeclareFieldFormat{url}{\newline\mkbibacro{URL}\addcolon\nobreakspace\url{#1}}
\usepackage{breakurl}
% \usepackage[automark]{scrlayer-scrpage}
\usepackage[left=2.2cm, right=1.5cm, top=3cm, bottom=4.5cm]{geometry}
\usepackage{amsmath,amssymb,amsfonts,textcomp}  % Mathegeschichten und Symbole
\usepackage{wasysym,latexsym}      	% für \Smiley und \Frowny,...
\usepackage[right]{eurosym}			% für \euro{} oder \EUR{53,15} Opt. left/right, jenachdem, ob 53,15€ oder €53,15
\usepackage{makecell}
\usepackage{lipsum}
\usepackage{float}               	% einfacher Weg, um Abb. zu fixieren mit [H]
\usepackage{nonfloat}
%\usepackage[scanall]{psfrag}        % Ersetzen von Text in Abbildungen (geht *nicht* mit pdfLaTeX)
\usepackage{lscape, pdflscape}      % soll Tabellen im Landscape-Format im Reader
									% automatisch drehen, sonst wie landscape
\usepackage{overpic}				% reinschreiben in ein Bild, Koordinaten jeweils von 0-100
\usepackage{subfigure}				% \subfigure{\includegraphics[...]{...}}
									% \subfigure{\includegraphics[...]{...}} <- mehrere Abb. in einer figure
%\usepackage[shell]{gnuplottex}
%\usepackage{pdfsync}				% für das Springen aus dem PDF in den Editor, nicht mehrnotwendig
\usepackage{graphicx} 		        % ohne genauere Spezifizierung in eckigen Klammern
\usepackage{setspace}				% für Anderung des Zeilenabstandes

%\usepackage[basic]{circ}

\usepackage{listings}				% Einbinden von Quellcode
\usepackage{pst-pdf, pstcol, pst-osci}

\usepackage{epstopdf}				% Beim kompilieren wird on the fly ein 	*.pdf erstellt.

%\usepackage[thickspace,thickqspace,squaren]{SIunits}
\usepackage{siunitx}
\sisetup{locale = DE}

\usepackage{fancyhdr}
\usepackage{titlesec}
\usepackage{pdfpages}


\titleformat{\chapter}
{\normalfont\Huge\bfseries}{\thechapter}{1em}{}
\titleformat{\section}
{\normalfont\huge\bfseries}{\thesection}{1em}{}
\titleformat{\subsection}
{\normalfont\LARGE\bfseries}{\thesubsection}{1em}{}
\titleformat{\subsubsection}
{\normalfont\LARGE\bfseries}{\thesubsubsection}{1em}{}


%---------------------------------- EINTRAEGE fuer  LAYOUT (evtl. anpassen)   -----------------------------------
% Disable single lines at the start of a paragraph (Schusterjungen)
\clubpenalty = 10000
%
% Disable single lines at the end of a paragraph (Hurenkinder)
\widowpenalty = 10000 \displaywidowpenalty = 10000
%

%-------------------------------------------------------------------------------
% Erstellt die Header und Footer der Seiten mit Autor pro Seite
% Wenn nicht zweiseitig ([OL,OR], ...) anpassen
%-------------------------------------------------------------------------------
\fancypagestyle{alle}{
	\fancyhf{}
	\fancyhead[EL,OR]{\rightmark}
	\fancyfoot[OL,ER]{HTBL Hollabrunn}
	\fancyfoot[C]{\schuelerAlle}
	\fancyfoot[EL,OR]{\thepage}
	\renewcommand{\headrulewidth}{0.3pt}
	\renewcommand{\footrulewidth}{0.3pt}
}

\fancypagestyle{plain}[alle]{
	\fancyfoot[C]{}
}

\fancypagestyle{SchuelerA}[alle]{
	\fancyfoot[C]{\schuelerA}
}

\fancypagestyle{SchuelerB}[alle]{
	\fancyfoot[C]{\schuelerB}
}

\fancypagestyle{SchuelerC}[alle]{
	\fancyfoot[C]{\schuelerC}
}


\fancypagestyle{SchuelerD}[alle]{
	\fancyfoot[C]{\schuelerD}
}

\renewcommand{\chaptermark}[1]{\markright{\thechapter\ #1}}
\renewcommand{\sectionmark}[1]{\markright{\thesection\ #1}}
%\renewcommand{\subsectionmark}[1]{\markright{\thesubsection\ #1}}
%\renewcommand{\subsubsectionmark}[1]{\markright{\thesubsubsection\ #1}}

%\sectionfont{\fontsize{12}{15}\selectfont}
%\subsectionfont{\fontsize{12}{15}\selectfont}
%\subsubsectionfont{\fontsize{12}{15}\selectfont}
\renewcommand\lstlistlistingname{Code-Verweise}

\parindent0em							% Einzug bei neuen Absätzen
\parskip1mm								% Abstand zwischen Absätzen
\sloppy									% nicht soooo genau mit dem Blocksatz
% ---------------------------------------------------------------------------------------------------------------
\def\schuelerEmpty{~}

% ------------------------------------------------------------------------------------------------------